\documentclass{article}

\usepackage[left=1.5cm, right=1.5cm, top=3cm, bottom = 3cm]{geometry}

\usepackage{amsmath}
\usepackage{amsfonts}
\usepackage{amssymb}
\usepackage{graphicx}
\usepackage{float}
\usepackage{wrapfig}
\usepackage{latexsym}
\usepackage{hyperref}
\usepackage{feynmf}
\usepackage{exscale}
\usepackage{relsize}
\usepackage{bm}%bold math, for vector
\linespread{1.1}

%%%%%%%
%第2章习题安排:
%%%%%%%
%WF 1, 
%TH 5, 7


\author{SMTMS-at-ANYINT}
\title{\bf{Solutions to Tuckerman's Statistical Mechanics Theory and Molecular simulation}\\Chapter 2}

\begin{document}
	\maketitle
	\section*{Problem 2.1}
	\paragraph{a)}
	From the ideal gas law, eqn.(2.2.2), 
	$$PV - nRT = 0,$$
	the work done on the system during the isothermal expansion is simply
	$$ \Delta W = -\int^{V_{2}}_{V_{1}} PdV = -\int^{V_{2}}_{V_{1}} \frac{nRT}{V} dV = -nRT\ln \frac{V_{2}}{V_{1}} = -\frac{N}{N_{0}}RT\ln \frac{V_{2}}{V_{1}}, $$ 
	\paragraph{b)}
	Acoording to Boltzmann's relation, equ.(3.2.11):
	$$S = k \ln \Omega, $$
	the change of entropy fro the isothermal process in part a is
	$$\Delta S = S_{2} - S_{1} = k \ln \Omega_{2} - k \ln \Omega_{1} = k\ln \frac{\Omega_{2}}{\Omega_{1}},$$
	as thetotalnumber of microscopic states available to the gas can be shown to be $\Omega \propto V^{N}(kT)^{3N/2}$, we can get
	$$\Delta S = k \ln \frac{\Omega_{2}}{\Omega_{1}} = k\ln \frac{V^{N}_{2}}{V^{N}_{1}} = Nk\ln\frac{V_{2}}{V_{1}}$$
	~\\
	Acoording to the first law of thermodynamics, equ.(2.2.5):
	$$\Delta E = \Delta Q + \Delta W,$$
	the heat absorbed in the isothermal process (which $\Delta E = 0 $) in part a is:
	$$\Delta Q = Q_{2} - Q_{1} = - \Delta W = \frac{N}{N_{0}}RT\ln \frac{V_{2}}{V_{1}} $$ 
	Acorrding to eqn. (2.2.19), the entropy of the system $\Delta S $ defined by
	
	$$\Delta S = \int_{1}^{2} \frac{dQ_{rev}}{T},$$
	so we can calculate the change of entropy 
	$$\Delta S = \frac{1}{T} \Delta Q = \frac{1}{T} \frac{N}{N_{0}}RT\ln \frac{V_{2}}{V_{1}} = \frac{N}{N_{0}}R\ln \frac{V_{2}}{V_{1}}, $$
	as $k=R/N_{0}$, we can get 
	$$\Delta S = Nk \ln \frac{V_{2}}{V_{1}}$$
	~\\
	It show that the two method yield the same entropy.
%	\paragraph{c)}
%	During

	
	
	
	
	
	
.
\end{document}